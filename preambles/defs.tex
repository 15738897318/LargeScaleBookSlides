\usepackage{graphicx,amsmath,amssymb,psfrag,mathtools}
\usepackage{soul}
\usepackage{amsmath,amsfonts,amsthm,bbm}
\usepackage{stmaryrd}
\usepackage{subcaption}



%Code block environment
\usepackage{listings}


\definecolor{lightgrey}{gray}{0.8}
\definecolor{medgrey}{gray}{0.6}
\definecolor{darkgrey}{gray}{0.4}
\usepackage{xcolor}
\lstset { %
    backgroundcolor=\color{black!5}, % set backgroundcolor
    basicstyle=\ttfamily,
    showstringspaces=false,
    commentstyle = \ttfamily,
    commentstyle=\color{commentgreen}\ttfamily,
    morecomment=[l][\color{darkgrey}]{//},
}



\usepackage{tikz}
\usetikzlibrary{matrix,chains,positioning,decorations.pathreplacing,arrows}
\usetikzlibrary{positioning,calc}
\usepackage{tkz-euclide}
%\usetkzobj{all}





%-------------------------------------------------------------------------------
%Definition of operator font
 \usepackage[bb=boondox]{mathalfa}
%% import \varmathbb without affecting other fonts
\usepackage{xparse}
\DeclareFontFamily{U}{ntxmia}{}
\DeclareFontShape{U}{ntxmia}{m}{it}{<-> ntxmia }{}
\DeclareFontShape{U}{ntxmia}{b}{it}{<-> ntxbmia }{}
\DeclareSymbolFont{lettersA}{U}{ntxmia}{m}{it}
\SetSymbolFont{lettersA}{bold}{U}{ntxmia}{b}{it}
\ExplSyntaxOn
\NewDocumentCommand{\varmathbb}{m}
 {
  \tl_map_inline:nn { #1 }
   {
    \use:c { varbb##1 }
   }
 }
\tl_map_inline:nn { ABCDEFGHIJKLMNOPQRSTUVWXYZ }
 {
  \exp_args:Nc \DeclareMathSymbol{varbb#1}{\mathord}{lettersA}{\int_eval:n { `#1+67 }}
 }
\exp_args:Nc \DeclareMathSymbol{varbbk}{\mathord}{lettersA}{169}
\ExplSyntaxOff
%%
\makeatletter
\DeclareFontFamily{U}{tipa}{}
\DeclareFontShape{U}{tipa}{m}{n}{<->tipa10}{}
\newcommand{\arc@char}{{\usefont{U}{tipa}{m}{n}\symbol{62}}}%

\newcommand{\arc}[1]{\mathpalette\arc@arc{#1}}

\newcommand{\arc@arc}[2]{%
  \sbox0{$\m@th#1#2$}%
  \vbox{
    \hbox{\resizebox{\wd0}{\height}{\arc@char}}
    \nointerlineskip
    \box0
  }%
}
\makeatother
\newcommand{\opA}{{\varmathbb{A}}}
\newcommand{\opB}{{\varmathbb{B}}}
\newcommand{\opC}{{\varmathbb{C}}}
\newcommand{\opD}{{\varmathbb{D}}}
\newcommand{\opE}{{\varmathbb{E}}}
\newcommand{\opF}{{\varmathbb{F}}}
\newcommand{\opG}{{\varmathbb{G}}}
\newcommand{\opH}{{\varmathbb{H}}}
\newcommand{\opI}{{\varmathbb{I}}}
\newcommand{\opJ}{{\varmathbb{J}}}
\newcommand{\opK}{{\varmathbb{K}}}
\newcommand{\opL}{{\varmathbb{L}}}
\newcommand{\opM}{{\varmathbb{M}}}
\newcommand{\opN}{{\varmathbb{N}}}
\newcommand{\opO}{{\varmathbb{O}}}
\newcommand{\opP}{{\varmathbb{P}}}
\newcommand{\opQ}{{\varmathbb{Q}}}
\newcommand{\opR}{{\varmathbb{R}}}
\newcommand{\opS}{{\varmathbb{S}}}
\newcommand{\opT}{{\varmathbb{T}}}
\newcommand{\opU}{{\varmathbb{U}}}
\newcommand{\opV}{{\varmathbb{V}}}
\newcommand{\opW}{{\varmathbb{W}}}
\newcommand{\opX}{{\varmathbb{X}}}
\newcommand{\opY}{{\varmathbb{Y}}}
\newcommand{\opZ}{{\varmathbb{Z}}}
\newcommand{\opZer}{\mathbb{0}}
%-------------------------------------------------------------------------------



%-------------------------------------------------------------------------------
%Definition of other font types
\newcommand{\va}{{\mathbf{a}}}
\newcommand{\vb}{{\mathbf{b}}}
\newcommand{\vc}{{\mathbf{c}}}
\newcommand{\vd}{{\mathbf{d}}}
\newcommand{\ve}{{\mathbf{e}}}
\newcommand{\vf}{{\mathbf{f}}}
\newcommand{\vg}{{\mathbf{g}}}
\newcommand{\vh}{{\mathbf{h}}}
\newcommand{\vi}{{\mathbf{i}}}
\newcommand{\vj}{{\mathbf{j}}}
\newcommand{\vk}{{\mathbf{k}}}
\newcommand{\vl}{{\mathbf{l}}}
\newcommand{\vm}{{\mathbf{m}}}
\newcommand{\vn}{{\mathbf{n}}}
\newcommand{\vo}{{\mathbf{o}}}
\newcommand{\vp}{{\mathbf{p}}}
\newcommand{\vq}{{\mathbf{q}}}
\newcommand{\vr}{{\mathbf{r}}}
\newcommand{\vs}{{\mathbf{s}}}
\newcommand{\vt}{{\mathbf{t}}}
\newcommand{\vu}{{\mathbf{u}}}
\newcommand{\vv}{{\mathbf{v}}}
\newcommand{\vw}{{\mathbf{w}}}
\newcommand{\vx}{{\mathbf{x}}}
\newcommand{\vy}{{\mathbf{y}}}
\newcommand{\vz}{{\mathbf{z}}}

\newcommand{\vA}{{\mathbf{A}}}
\newcommand{\vB}{{\mathbf{B}}}
\newcommand{\vC}{{\mathbf{C}}}
\newcommand{\vD}{{\mathbf{D}}}
\newcommand{\vE}{{\mathbf{E}}}
\newcommand{\vF}{{\mathbf{F}}}
\newcommand{\vG}{{\mathbf{G}}}
\newcommand{\vH}{{\mathbf{H}}}
\newcommand{\vI}{{\mathbf{I}}}
\newcommand{\vJ}{{\mathbf{J}}}
\newcommand{\vK}{{\mathbf{K}}}
\newcommand{\vL}{{\mathbf{L}}}
\newcommand{\vM}{{\mathbf{M}}}
\newcommand{\vN}{{\mathbf{N}}}
\newcommand{\vO}{{\mathbf{O}}}
\newcommand{\vP}{{\mathbf{P}}}
\newcommand{\vQ}{{\mathbf{Q}}}
\newcommand{\vR}{{\mathbf{R}}}
\newcommand{\vS}{{\mathbf{S}}}
\newcommand{\vT}{{\mathbf{T}}}
\newcommand{\vU}{{\mathbf{U}}}
\newcommand{\vV}{{\mathbf{V}}}
\newcommand{\vW}{{\mathbf{W}}}
\newcommand{\vX}{{\mathbf{X}}}
\newcommand{\vY}{{\mathbf{Y}}}
\newcommand{\vZ}{{\mathbf{Z}}}

\newcommand{\cA}{{\mathcal{A}}}
\newcommand{\cB}{{\mathcal{B}}}
\newcommand{\cC}{{\mathcal{C}}}
\newcommand{\cD}{{\mathcal{D}}}
\newcommand{\cE}{{\mathcal{E}}}
\newcommand{\cF}{{\mathcal{F}}}
\newcommand{\cG}{{\mathcal{G}}}
\newcommand{\cH}{{\mathcal{H}}}
\newcommand{\cI}{{\mathcal{I}}}
\newcommand{\cJ}{{\mathcal{J}}}
\newcommand{\cK}{{\mathcal{K}}}
\newcommand{\cL}{{\mathcal{L}}}
\newcommand{\cM}{{\mathcal{M}}}
\newcommand{\cN}{{\mathcal{N}}}
\newcommand{\cO}{{\mathcal{O}}}
\newcommand{\cP}{{\mathcal{P}}}
\newcommand{\cQ}{{\mathcal{Q}}}
\newcommand{\cR}{{\mathcal{R}}}
\newcommand{\cS}{{\mathcal{S}}}
\newcommand{\cT}{{\mathcal{T}}}
\newcommand{\cU}{{\mathcal{U}}}
\newcommand{\cV}{{\mathcal{V}}}
\newcommand{\cW}{{\mathcal{W}}}
\newcommand{\cX}{{\mathcal{X}}}
\newcommand{\cY}{{\mathcal{Y}}}
\newcommand{\cZ}{{\mathcal{Z}}}
%-------------------------------------------------------------------------------




%-------------------------------------------------------------------------------
%% macros for math notions and operators
\newcommand{\EE}{{\mathbb{E}}}
\newcommand{\expec}{\mathbb{E}}
\newcommand{\Prob}{{\mathrm{Prob}}} % probability

\newcommand{\reals}{\mathbb{R}}
\newcommand{\RR}{\mathbb{R}}
\newcommand{\complex}{\mathbb{C}}
\newcommand{\CC}{\mathbb{C}}
\newcommand{\nats}{\mathbb{N}}
\newcommand{\NN}{\mathbb{N}}
\newcommand{\ZZ}{\mathbb{Z}}
\newcommand{\bigO}{\mathcal{O}}
\newcommand{\order}[1]{{\mathcal{O}\left(#1\right)}}
\renewcommand{\SS}{{\mathbb{S}}}
\newcommand{\SSp}{\mathbb{S}_{+}}
\newcommand{\SSpp}{\mathbb{S}_{++}}
\newcommand{\sign}{\mathrm{sign}}
\newcommand{\vzero}{\mathbf{0}}
\newcommand{\vone}{{\mathbf{1}}}

\renewcommand{\Re}{\operatorname{Re}} 	%Real part
\renewcommand{\Im}{\operatorname{Im}}	%imaginary part

%\newcommand{\supp}{{\mathrm{supp}}} % support
\newcommand{\range}{\mathrm{range}\,} % domain
\newcommand{\tr}{{\mathrm{tr}}} % trace
%-------------------------------------------------------------------------------





%-------------------------------------------------------------------------------
%% Theorem definitions
\setbeamertemplate{theorems}[ams style] 
\newtheorem*{theorem*}{Theorem}
%\newtheorem{lemma}{Lemma}    % already provided by amsthm
\newtheorem{proposition}{Proposition}
%\newtheorem{proof}{Proof}  % already provided by amsthm


%-------------------------------------------------------------------------------
%% operator and convex analysis definitions

\newcommand*{\fix}{\mathrm{Fix}\,}
\newcommand*{\zer}{\mathrm{Zer}\,}
\newcommand*{\gra}{\mathrm{Gra}\,}
\newcommand{\prox}{\mathrm{Prox}}
\newcommand{\proj}{\Pi}
\newcommand{\aff}{\mathrm{aff}\,}    %affine hull
\newcommand{\intr}{\mathrm{int}\,}   %interior
\newcommand{\relint}{\mathrm{ri}\,}  %relative interior
\newcommand{\dom}{\mathrm{dom}\,} % domain
\newcommand{\epi}{\mathrm{epi}\,} % epigraph
\newcommand{\dist}{\mathrm{dist}}
\newcommand{\lagrange}{\mathbf{L}}  %saddle function
\newcommand{\fitzpatrick}{\mathbf{F}}   %Fitzpatrick function
\newcommand{\vecdelay}{\boldsymbol{d}}   %vector delay
\DeclareMathOperator*{\argmin}{argmin}
\DeclareMathOperator*{\argmax}{argmax}

%-------------------------------------------------------------------------------
%SRG definitions
\newcommand{\ereal}{\overline{\mathbb{R}^2}}
\newcommand{\ecomplex}{\overline{\mathbb{C}}}
\newcommand{\binfty}{{\boldsymbol \infty}}
\newcommand{\rarc}{\mathrm{Arc}^+}
\newcommand{\larc}{\mathrm{Arc}^-}


%-------------------------------------------------------------------------------




%-------------------------------------------------------------------------------
%Miscellaneous Stuff
%% sequences
\newcommand{\itom}{_{i=1}^{m}}
\newcommand{\ieqm}{i=1,\dots,m}

% use \numberthis to add an equation number in align*
\newcommand\numberthis{\addtocounter{equation}{1}\tag{\theequation}}

\newcolumntype{P}[1]{>{\centering\arraybackslash}p{#1}}
